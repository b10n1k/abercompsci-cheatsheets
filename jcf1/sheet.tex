\documentclass[twocolumn]{article}
\usepackage[hmargin={0.5in,0.5in},vmargin={0.5in,0.5in}]{geometry}
\usepackage{fancyvrb}
\newcommand{\HRule}{\noindent\rule{\linewidth}{0.1mm}\\}
\newcommand{\blankline}{\vspace{1ex}}
\newcommand{\subscript}[1]{\ensuremath{_{\tiny\textrm{#1}}}}
\newcommand{\superscript}[1]{\ensuremath{^{\tiny\textrm{#1}}}}
\newcommand{\jref}[2]{\textsc{#1}\subscript{\pageref{#2}}}
\newcommand{\orn}{\begin{center}$$\bullet$$\end{center}}

\DefineVerbatimEnvironment{vv}{Verbatim}{
    %numbers=left,numbersep=5pt,
    %frame=lines,framerule=0.5mm,
    fontsize=\small,xleftmargin=15pt}
\DefineVerbatimEnvironment{v}{Verbatim}{
    %numbers=left,numbersep=5pt,
    %frame=lines,framerule=0.5mm,
    fontsize=\small,xleftmargin=15pt}

\DefineVerbatimEnvironment{vv2}{Verbatim}{
    %numbers=left,numbersep=5pt,
    %frame=lines,framerule=0.5mm,
    fontsize=\scriptsize,xleftmargin=15pt}

\begin{document}
\section{Directories and files}
Directories in UNIX are like folders in Windows. Every user has their
home directory, and at any given time they have a working directory --- the
directory in which they're currently working. It's sometimes called
the current directory or current working directory.

\subsection{Home and working directories}
When you log into a UNIX system, you start out in your home directory. Your
UNIX prompt will show this:
\begin{v}
central:~ $
\end{v}
The \verb+central+  is the name of the machine you are logged into, and
the \verb+~+  is the name of your home directory. This is just a shorthand:
if you use the \verb+pwd+ command (Print Working Directory) you'll see
the full name. In my case, this happens:
\begin{v}[commandchars=\\\[\]]
central:~ $ \emph[pwd]
/aber/jaf18
\end{v}
(I'm using italics to show the command I typed in, rather than Central's
output.)

\subsection{Paths}
Just like folders in Windows, directories can contain files and other directories.
We talk about a \emph{path} to a file or directory in UNIX --- it specifies a file by
showing how to get there either from the current working directory (in which
case it's a \emph{relative path}) or from the system's top level directory
(the \emph{root} directory,) in which case it's a \emph{absolute path}.

The path is made up of directory names, separated by slashes, until you
get to the file or directory you're talking about.  Each
directory is inside the previous directory in the path.

As a silly example, if planet Earth were a UNIX file system, an
absolute path to the tutorial room might be
\begin{v}
/earth/gb/wales/aberystwyth/penglais/llandinam/c57a
\end{v}

If my ``working directory'' were \verb+aberystwyth+, a relative path ---
which shows how to get to another directory from the working directory ---
might be
\begin{v}
penglais/llandinam/c57a
\end{v}
More realistically, a file called \verb+project1.tex+ inside a directory
called \verb+work+ in my home directory would have the absolute path
\begin{v}
/aber/jaf18/work/project1.tex
\end{v}
or, using the shorthand \verb+~+ character\footnote{called a ``tilde''}:
\begin{v}
~/work/project1.tex
\end{v}
or, if we want to look at the home directory of a different user than the
one currently logged in,
\begin{v}
~jaf18/work/project1.tex
\end{v}
Relative paths always start with a directory or file name, while
absolute paths always start with a ``/'' or ``~''.

\subsubsection{Going up the tree}
Let's look at the silly example of the ``Earth file system'' again. If
I were in (say) Physics Main, how could I get to C57A? In other
words, what's the \emph{relative path} from there to C57A?

Firstly, let's find out the absolute path to Physics Main:
\begin{v}
/earth/gb/wales/aberystwyth/penglais/physics/physicsMain
\end{v}
Now we can see what we need to do:
\begin{itemize}
\item go up to the \verb+physics+ directory
\item go up again to the \verb+aberystwyth + directory
\item go down to the \verb+llandinam+ directory
\item go down to \verb+c57a+
\end{itemize}
We can go ``up'' a level inside a path by using ``..'' instead of a directory
name:
\begin{v}
../../llandinam/c57a
\end{v}
You'll do this a lot --- for example, you might do some work inside your
\verb+work/project1+ directory, and then want to do some more work on
Project 2 in \verb+work/project2+. That might go something like this:
\begin{v}[commandchars=\\\[\]]
central:~ $ \emph[cd work/project1]
central:~/work/project1 $
...
central:~/work/project1 $ \emph[cd ../project2]
central:~/work/project2 $ 
\end{v}
As well as the ``..'' shorthand for ``the directory above,'' (also called
the \emph{parent} directory) we also
have ``.'' for ``the working directory.'' This can be useful in move 
and copy commands. 

\section{The anatomy of a command}
Commands typically have three parts:
\begin{itemize}
\item The name of the command
\item The options, which start with dashes
\item The parameters of the command
\end{itemize}
For example, the \verb+ls+ command lists files. It has a lot of options,
a couple of which are \verb+-l+ (to give a long listing) and \texttt{-a}
(to include hidden files.) Any parameters it takes are the names of files
and directories it should list. So
\begin{v}
ls -l -a ../project1 ../project2
\end{v}
will produce a long listing, containing all the hidden files, of 
the contents of the directories \verb+project1+ and \verb+project2+, both
of which are inside the current directory!

A lot of the time you can join options together, so 
\begin{v}
ls -la  ../project1 ../project2
\end{v}
will do the same thing.

\subsection{Wildcards}
Some commands can work on many files --- for example, the \verb+ls+ command
above. To help, we can use so-called ``wildcards'' which let us 
specify files given only a part of their name. The ``*'' wildcard matches
any sequence of characters, and the ``?'' wildcard matches any single
character. So:
\begin{v}
ls a*
\end{v}
means ``list all the files whose names are `a' followed by any sequence 
of characters'' --- in other words, all the files whose names start with
``a''. Similarly,
\begin{v}
ls *a
\end{v}
will list all the files whose names \emph{end} with the letter ``a''.
Here's a more complex one:
\begin{v}
ls *.???
\end{v}
This will specify files which start with any sequence of characters,
followed by a ``.'', followed by exactly three characters. 

\section{The most important command}
\begin{tabular}{l|p{2in}}
man \emph{command} & get information about a command
\end{tabular}


\section{Changing directory}
\begin{tabular}{l|p{2in}}
cd \emph{path} & change the working directory to a specified directory. Some
examples follow.\\
cd ~ & change to your home directory  \\
cd & shorthand way to change to your home directory \\
cd .. & change to the directory within which your current directory 
is contained (the \emph{parent} directory)\\
cd / & change to the root directory of the whole system\\
cd . & change the current working directory to the current working
directory! (Yes, it's pointless.) \hrule \\
pwd & print the absolute path of the current working directory
\end{tabular}

\section{Listing directories}
\begin{tabular}{l|p{2in}}
ls & get a short listing of the contents of the current working directory\\
ls \emph{path} & get a short listing of the contents of a named directory\\
ls .. & get a short listing of the contents of the parent directory\\
ls -l & get a detailed listing of the contents of the current working
directory\\
ls -ld & get a detailed description of the current working directory itself\\
ls -l \emph{path} & get a detailed listing of the contents of a named directory\\
ls -la \emph{path} & get a detailed listing of the contents of a named directory, including those files which are usually hidden.\\
ls -la \emph{path}|more & as above, but split into pages (see Pipes, below)
\end{tabular}

\section{Copying files}
\begin{tabular}{l|p{2in}}
cp \emph{file1} \emph{file2} & makes a new copy of \emph{file1} called \emph{file2}, keeping the old one.
If there's already a file called \emph{file2} it will be overwritten.\\
cp \emph{file} \emph{directory} & makes a new copy of \emph{file} inside
\emph{directory}, but with the same name (so it'll be \emph{directory/file} .
If there's already a file of that name inside \emph{directory} it will
be overwritten.\\
cp \emph{file1} \emph{directory/file2} & makes a new copy of \emph{file1}
inside \emph{directory}, called \emph{file2} (so it'll be \emph{directory/file2} .
If there's already a file of that name inside \emph{directory} it will
be overwritten.\\
cp \emph{list-of-files} \emph{directory} & copy all the files in the list
into a directory.
\end{tabular}

\subsubsection{Examples}
\begin{tabular}{l|p{2in}}
cp foo bar & make a copy of the file \emph{foo} called \emph{bar}, overwriting
any file of that name which may already exist.\\
cp foo .. & copy the file \emph{foo} into the parent directory.\\
cp ../foo . & copy the file \emph{foo}, which is in the parent directory, into the current working directory \\
\end{tabular}

\subsection{Moving files}
The \verb+mv+ command works almost exactly the same way as \verb+cp+, but
it removes the original file!
\begin{tabular}{l|p{2in}}
mv \emph{file1} \emph{file2} & changes the name of \emph{file1} to \emph{file2}, keeping the old one.
If there's already a file called \emph{file2} it will be overwritten.\\
mv \emph{file} \emph{directory} & moves \emph{file} into
\emph{directory}, so it'll now be \emph{directory/file} .
If there's already a file of that name inside \emph{directory} it will
be overwritten.\\
mv \emph{file1} \emph{directory/file2} & moves \emph{file1}
into \emph{directory}, and changes the name to \emph{file2} (so it'll be \emph{directory/file2} .
If there's already a file of that name inside \emph{directory} it will
be overwritten.\\
mv \emph{list-of-files} \emph{directory} & move all the files in the list
into a directory.
\end{tabular}

\subsection{Managing directories and deleting stuff}
\begin{tabular}{l|p{2in}}
mkdir \emph{path} & will make a new directory with that path\\
rmdir \emph{path} & will remove the directory with that path\\
rm \emph{list-of-files} & will remove some files\\
rm -r \emph{list-of-paths} & will remove the named files and directories
\emph{recursively} --- that is, it will also remove the files and directories
any directories contain. \textbf{Use with extreme caution.} \\
rm -rf \emph{list-of-paths} & will remove the named files and directories
\emph{recursively} --- that is, it will also remove the files and directories
any directories contain \emph{and not ask you to confirm!}
\textbf{Use with even more extreme caution. You WILL do this wrong one day.}\\
rm -rf * & \textbf{never use this} --- it will silently destroy everything
in the current working directory. I shouldn't even be telling you about it.
\end{tabular}

\subsection{Pipes and redirection}
Usually, input for a command comes from the keyboard and output goes
to the screen. Sometimes it's useful to change this --- to get input
from a file and output to a file. To do this, use the ``\textless'' character
to get input from a file (think of the arrow as pointing towards the command
name) and ``\textgreater'' to send output to a file.

For example, we an put a long listing of our home directory into a file:
\begin{v}
ls -l >big_listing
\end{v}

We can also send the output of one command into the input of another using
a \emph{pipe.} For example, the \verb+more+ command takes its input and
splits it up into pages. If we have a big directory, we can list it
page by page using:
\begin{v}
ls -l | more
\end{v}
This takes the output of \texttt{ls} and pipes it into \texttt{more}, which
splits it into pages. Another example:
\begin{v}
ps ax | grep MyJavaProgram
\end{v}
This gets a list of all the running programs on the system, and searches
through that list for lines containing the string ``MyJavaProgram''.
Useful when a program goes out of control and you need to kill it!

A fun example:
\begin{v}
ps ax | cut -c 28- | sort | uniq | wc -l
\end{v}
This
\begin{itemize}
\item list all processes
\item cuts out the first 27 characters from each line, so we only
get the program names
\item sorts them alphabetically
\item removes duplicate adjacent lines
\item counts the number of lines
\end{itemize}
So we get a count of all the uniquely named processes running on the system.


\subsection{Showing the contents of files}
\begin{tabular}{l|p{2in}}
cat \emph{list-of-files} & just joins all the files together and outputs
them \\
more \emph{list-of-files} & shows the files page by page \\
more & splits the input into pages (input comes from the keyboard by
default, but you'll typically redirect it from another command using a pipe)
\end{tabular}


\subsection{Permissions}
Each UNIX file has a set of permissions describing who can do what to it.
There are three types of permission:
\begin{itemize}
\item read permission --- if you have this, you can see the contents of the
file or directory;
\item write permission --- if you have this, you can write to the file or
add new files (or delete or move) in the directory;
\item execute permission --- if you have this permission on a
program file or script you can run it; if you have it on a directory you
can \verb+cd+ into it (make it your working directory.)
\end{itemize}
The permissions can be set differently for three different types of person:
\begin{itemize}
\item the owner of the file --- whoever originally created it
\item the group ID of the file --- a rather crude way of setting things up
so groups of people can have special access to a file; I won't worry about it too much
\item everyone else
\end{itemize}
\subsubsection{The output of ls}
These permissions are listed by \verb+ls -l+ in a very terse way:
\begin{v}[commandchars=\\\[\]]
central:~ $ \emph[ls -l sheet.pdf]
-rw-rw-r--. 1 jaf18 jaf18 70273 Nov  8 14:05 sheet.pdf
\end{v}
The elements in order are:
\begin{itemize}
\item permissions block \verb+-rw-rw-r--.+ 
\item the ``inode count'' (which is typically 1 for a file)
\item the owner's user ID
\item the group's user ID
\item the size in bytes
\item last modified date and time
\item name
\end{itemize}
Looking at the permissions block we can see that it's split into
elements:

\begin{tabular}{l p{2in}}
\verb+-+ & the type of the file: ``d'' would be a directory\\
\verb+rw-+ & permissions for the user themselves\\
\verb+rw-+ & permissions for the group\\
\verb+r--+ & permissions for other users
\end{tabular}

So we can see that I (and anyone in my group, which is only me) am allowed to 
read and write the file, and other people can read it.

If I look at my home directory, things are different:
\begin{v}[commandchars=\\\[\]]
central:~ $ \emph[ls -ld ~]
drwx------. 85 jaf18 jaf18 4096 Nov  8 13:23 /aber/jaf18
\end{v}
Note that I'm using the \verb+-d+ option to look at the actual directory:
if I'd not used it, I'd get the contents of the directory!

Here, the ``inode count'' shows the number of files in the directory,
but what we're really interested in is the permissions:

\begin{tabular}{l p{2in}}
\verb+d+ & indicates this is a directory\\
\verb+rwx+ & permissions for the user themselves\\
\verb+---+ & permissions for the group\\
\verb+---+ & permissions for anyone else
\end{tabular}

So I can read, write and \texttt{cd} into the directory and no-one
else can do anything at all. Which is appropriate.

\subsubsection{Changing permissions the ``easy'' way}
\begin{tabular}{l|p{2in}}
chmod o+x \emph{list-of-paths} & add execute permissions for other users to some files or directories\\
chmod o-x \emph{list-of-paths} & remove execute permissions for other users from some files or directories\\
chmod a+rw \emph{list-of-paths} & add read and write permissions for all users (that's user, group and other)\\
chmod o-rwx \emph{list-of-paths} & remove all permissions for other users\\
chmod u+rx \emph{list-of-paths} & add execute and read permissions for myself\\
chmod g-x \emph{list-of-paths} & remove execute permissions for the group
\end{tabular}
You should get the idea: the first letter is ``u'' for the user, ``g'' for
the group and ``o'' for others; the second character is ``+'' to add
permissions and ``-'' to remove them; and the final set of characters is
the permissions to add or remove. There's more it can do --- run
\verb+man chmod+ to get the full info.

\subsubsection{Changing permissions the hard way}
This involves encoding the \texttt{rwxrwxrwx} block --- which divides into
user, group and other --- as an octal number! 
\begin{tabular}{l|p{2in}}
chmod 755 \emph{list-of-paths} & set the permissions to: user can read, write
and execute; group and read and execute; others can read and execute.
\end{tabular}
You might want to read up on this yourself, or stick with the other technique.
I tend to use both.

\section{Other useful commands (and keys)}

\begin{tabular}{l|p{2in}}
logout & logs you out of the computer\\
touch \emph{file} & creates an empty file, or updates the last-modified
date on an existing file\\
grep \emph{pattern} \emph{files} & look for a ``pattern'' inside files --- 
you'll be covering these patterns (called \emph{regular expressions}) later on,
but it will also work on simple strings\\
sort & sorts input into alphabetical order and outputs it \\
du -H . & shows disk usage in current directory (find out what the H does!)\\
w & tells you who's logged in and what they're up to \hrule\\
\emph{(up arrow)} & shows previous commands\\
\emph{(tab key)} & autocomplete\\

\end{tabular}




\end{document}


